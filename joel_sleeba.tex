\PassOptionsToPackage{pdfpagelabels=false}{hyperref}
\documentclass[11pt,a4paper,serif]{moderncv}
\moderncvstyle{classic}
\moderncvcolor{blue}
\usepackage[scale=0.75]{geometry}
\setlength{\footskip}{136.00005pt}
% \ifxetexorluatex
  % \usepackage{fontspec}
  % \usepackage{unicode-math}
  % \defaultfontfeatures{Ligatures=TeX}
  % \setmainfont{Noto Serif Display}
  % \setsansfont{Roboto}
  % \setmonofont{Noto Sans Mono}
  % \setmathfont{Noto Sans Math} 
% \else
  \usepackage[T1]{fontenc}
  \usepackage{lmodern}
% \fi

% \hypersetup{
%   colorlinks=true,
%   linkcolor='blue',
%   filecolor='magenta',
%   urlcolor='cyan',
%   }
% \urlstyle{same}

\name{Joel}{Sleeba}
\email{sleeba21@iisertvm.ac.in}
% \email{joelsleeba24@gmail.com}
\homepage{joelsleeba.github.io}
\social[twitter]{JoelSleeba}
\social[github]{joelsleeba}

\begin{document}
\makecvtitle

\section{About Me}
I am a $2^{nd}$ year masters student in mathematics at the Indian Institute of Science Education and Research, Thiruvananthapuram and I am doing my master's project on \textbf{Fourier analysis} under the guidance of Dr. P Devaraj. I am broadly interested in \textbf{Harmonic Analysis and Operator Theory}. I am also passionate about open source software and a proud Linux user. 

\section{Education}
  \cventry{2021--present}{M.Sc Mathematics}{IISER Thiruvananthapuram}{CGPA: 8.06}{}{
    \begin{itemize}
      \item \textbf{Relevant courses:} Functional Analysis, Measure Theory, Topology, Abstract Algebra, Analysis on Manifolds, Finite Frames, Represenatation Theory
      \item \textbf{Readings} 
        \begin{itemize}
          \item \emph{Functional Analysis: A First Course}, S. Kumaresan and D. Sukumar
          \item \emph{Measure, Integration and Real Analysis}, Sheldon Axler 
          \item \emph{Topology}, James Munkres
          \item \emph{Abstract Algebra}, David S. Dummit, Richard M. Foote
          \item \emph{Analysis on Manifolds}, James Munkres
          \item \emph{Finite Frames}, Peter G. Casazza, Gitta Kutynoik
          \item \emph{Complex Analysis}, Elias M. Stein, Rami Shakarchi
        \end{itemize}
    \end{itemize}
  }
  \cventry{2018--2021}{B.Sc Mathematics}{Madras Christian College}{Chennai}{GPA: 8.03}{
    \begin{itemize}
      \item \textbf{Relevant courses:} Real Analysis, Algebra, Linear Algebra, Number Theory, Discrete Mathematics
      \item \textbf{Readings}
        \begin{itemize}
          \item \emph{Understanding Analysis}, Stephen Abbott
          \item \emph{Topics in Algebra}, I. N. Herstein
          \item \emph{Linear Algebra Done Right}, Sheldon Axler
          \item \emph{Elementary Number Theory}, David M. Burton
        \end{itemize}
    \end{itemize}
  }

\section{Master's Thesis}
\cvitem{Title}{\emph{To be determined}}
\cvitem{Supervisor}{Dr. P Devaraj}
\cvitem{Description}{
  \begin{itemize}
    \item \textbf{Current reading:} Classical Paley Wiener theorems from Chapter 19, Holomorphic Fourier Transforms of \emph{Real and Complex Analysis} by Walter Rudin
    \item \textbf{Previous reading:} Fourier series on a circle and Fourier transforms from \emph{Early Fourier Analysis} by Hugh L. Montgomery
  \end{itemize}
  }

\newpage
\section{Summer Schools}
  \cventry{2021 April -- 2021 May}{Mathematics Training and Talent Search}{Online}{}{}{Participated in level 1 of the national annual summer camp hosted by MTTS trust.
  \textbf{Topics}
  \begin{itemize}
    \item \emph{Foundations}, S. Kumaresan
    \item \emph{Algebra}, Krishna Hanumanthu
    \item \emph{Analysis}, G. Santhanam
    \item \emph{Topology}, Pratulnanda Das
  \end{itemize}
}

  \cventry{2020 May -- 2020 June}{Mathematics Training and Talent Search}{Online}{}{}{Participated in level 0 of the national annual summer camp hosted by MTTS trust.
  \textbf{Topics}
  \begin{itemize}
    \item \emph{Foundations}, S. Kumaresan
    \item \emph{Algebra}, H. Ananthnarayan
    \item \emph{Analysis}, A. J. Jayanthan
  \end{itemize}
}  

\section{Achievements}
  \cventry{2021}{Rank 1}{M.Sc mathematics entrance examination}{Pondicherry University}{}{}
  \cventry{2021}{Rank 3}{M.Sc mathematics entrance examination}{Hyderabad Central University}{}{}

\section{Skills}
  \cventry{Scripting}{Python, Bash}{}{}{}{
    \textit{Python Libraries:} Matplotlib, Scikit, Numpy \\ 
    \textit{Bash:} Task automation in Linux}
  \cventry{Programming}{C, C++, Java}{}{}{}{
    \textit{C:} Algorithms involving pointers and structures \\ 
    \textit{C++, Java:} Object Oriented Programming paradigm}
  \cventry{CAS}{MATLAB, GNU Octave, Maxima, SageMath}{}{}{}{
    \textit{MATLAB,Maxima}: Basics as part of Coursework. \\ 
    \textit{GNU Octave, SageMath}: Introductory knowledge}
  \cventry{Markup}{$LaTeX$, Markdown, HTML}{}{}{}{}
  \cventry{Languages}{English, Malayalam, Tamil, Hindi}{}{}{}{Native proficiency in English and Malayalam. Elementary proficiencty in Tamil and Hindi.}

\section{Additional Courses}
  \cventry{2020 July -- 2020 Dec}{CS101.2x: Object-Oriented Programming}{IITBombayX}{MOOC}{A+}{
    \textit{Instructors:} \href{https://www.cse.iitb.ac.in/~dbp/}{Prof. Deepak B Phatak}, \href{https://www.cse.iitb.ac.in/~supratik/}{Prof. Supratik Chakraborty}\\
    \textit{Topics:} Object Oriented Paradigm, Algorithms in OOP
  }
  \cventry{2020 July -- 2020 Dec}{CS101.1x: Programming Basics}{IITBombayX}{MOOC}{A+}{
    \textit{Instructors:} \href{https://www.cse.iitb.ac.in/~dbp/}{Prof. Deepak B Phatak}, \href{https://www.cse.iitb.ac.in/~supratik/}{Prof. Supratik Chakraborty}\\
    \textit{Topics:} Syntax and basic algorithms in C++
  }

\section{Teaching}
  \cventry{2021 Sep -- 2021 Oct}{Mentor}{Online Foundation Course in Mathematics}{MTTS Trust}{}{Cleared doubts and guided discussions for first and second year undergraduates.}


\section{Projects}
  \cventry{2022 Sep -- 2022 Nov}{Math Modelling}{The International Genetically Engineered Machine competition (iGEM)}{IISER Thiruvananthapuram}{}{Modelled vesicle internalization for a breast cancer drug delivery system. The model can be accessed \href{https://2022.igem.wiki/iiser-tvm/model}{here.}}
  \cventry{2022 Sep -- 2022 Nov}{Web Developement}{The International Genetically Engineered Machine competition (iGEM)}{IISER Thiruvananthapuram}{}{Developed the team webpage. The website can be accessed \href{https://2022.igem.wiki/iiser-tvm/}{here.}}
  \cventry{2021 January}{XOR encryptor}{}{}{}{Developed a python script that can encrpyt any file or text using a key or password. The repository can be accessed from \href{https://github.com/joelsleeba/xor-encryptor}{here.}}
  \cventry{2020 May}{CSSart}{}{}{}{Developed a series of websites using CSS and HTML. The website can be accessed \href{https://joelsleeba.github.io/CSSart}{here.}}

\end{document}

